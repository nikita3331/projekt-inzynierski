\section{Zawartość pracy}
Pierwszy rozdział opisuje cele oraz założenia pracy.Dokonano gruntownej analizy problemu, który zostanie rozwiązany w dalszym ciągu pracy.

W drugim rozdziale wykonano przegląd istniejących metod mających na celu generację trójwymiarowych obiektów na podstawie danych z kamery głębi. Dokonano porównania pomiędzy dostępnymi na rynku skanerami 3D bazującymi na różnych technologiach pomiarowych. Wymieniono ich parametry techniczne. Zobrazowano w jakich warunkach dana metoda pomiarowa powinna zostać wykorzystana. Ukazane zostały również technologie jakimi posługiwano się w przeszłości do generacji trójwymiarowych modeli. Na koniec przedstawione zostały zastosowania współczesnych skanerów 3D.

W kolejnym rozdziale przedstawiony jest model oraz konstrukcja skanera 3D. Wyjaśniono metody służące do przetworzenia danych uzyskanych z kamery głębi w chmurę punktów. Przedstawiono koncepcje istniejących rozwiązań służących do rekonstrukcji powierzchni oraz kształtu obiektów z chmury punktów.

Czwarty rozdział przedstawia podsumowanie zarówno wykonanej pracy, jak i otrzymanych efektów. Ukazane zostaną również metody analizy oraz obróbki danych, które mają posłużyć do cyfrowej implementacji rzeczywistych obiektów zarejestrowanych przez kamerę RGBD. Przedstawiono opisy zastosowanych algorytmów oraz kolejność ich wykonywania na podstawie autorskiego programu w języku Python. Poddano analizie pod względem dokładności rezultaty pomiarów w porównaniu do rzeczywistych wartości mierzonych.

Ostatni rozdział porusza kwestię potencjalnych możliwości udoskonalenia urządzenia.