\section{Rekonstrukcja powierzchni}
Wszystkie powyższe metody pozwalały na określenie chmury punktów danego obiektu. W celu wyeksportowania gotowego modelu, potrzebna jest również znajomość całej powierzchni danego przedmiotu. W tym pomocne okazują się różne metody rekonstrukcji powierzchni. Jest to kluczowy element działania algorytmu wirtualizacji rzeczywistych obiektów do postaci modeli 3D. Powstało wiele prac naukowych na temat różnych metod rekonstrukcji powierzchni, poniżej zostały omówione najważniejsze z nich.
\subsection{Trójwymiarowa triangulacja Delaunaya}
Triangulacja powierzchni metodą Delaunaya jest jedną z popularniejszych metod nakładania siatki na nieregularnie rozprowadzone punkty. Polega ona na połączeniu punktów w nieprzecinające się trójkąty, dzięki czemu możliwe jest nałożenie na nie koloru oraz utworzenie meshu na danym obiekcie. Triangulację Delaunaya można wykonać dla dowolnej N-wymiarowej płaszczyzny, w danym przykładzie zostanie opisana metoda dla trójwymiarowej chmury punktów \cite{cignoni1998dewall}. Schemat działania tej metody został opisany poniżej. 
\begin{enumerate}
    \item Tworzony jest czworościan zawierający wszystkie punkty ze zbioru.
    \item Tworzona jest lista czworościanów do usunięcia oraz czworościanów do pozostawienia.
    \item Dla każdego punktu sprawdzane jest czy leży on wewnątrz sfery opisanej na czworościanie do pozostawienia. Jeśli leży,to tworzone są 4 czworościany z wierzchołków starego oraz z nowego punktu.
    \Item Na końcu początkowy czworościan wewnątrz którego znajdowały się wszystkie punkty zostaje usunięty. Skasowane zostają również wszystkie czworościany mające z nim wspólny wierzchołek.
\end{enumerate}
Zaletą tej metody jest to, że jest ona inkrementacyjna. Punkty do triangulacji są dodawane po kolei,a co za tym idzie, nie ma potrzeby tworzenia nowej siatki po dodaniu pojedynczego punktu. Wystarczy jedynie wykonać jedną fazę algorytmu i dopisać utworzone wierzchołki czworościanów do listy.
Modyfikacją tej metody jest triangulacja DeWall zaproponowana przez włoskich naukowców w 1997 roku \cite{cignoni1998dewall}. Jej nazwa pochodzi od Delaunay i ściana (ang. \textit{Wall}). Pojęcie ściany oznacza w tym wypadku rekurencyjne dzielenie zbioru punktów na dwa podobne do siebie obszary, następnie równoległe utworzenie dwóch niezależnych od siebie triangulacji Delaunaya. Dzięki podzieleniu zbioru punktów na niezależne od siebie obszary, możliwe staje się wykorzystanie wielowątkowości procesora do obliczeń równoległych. To znacząco przyspiesza pracę algorytmu.
\subsection{Algorytm maszerujących sześcianów}
Algorytm maszerujących sześcianów jest metodą typu dziel i podbijaj zaproponowaną w 1987 roku przez amerykańskich naukowców w Nowym Yorku \cite{lorensen1987marching}. W przeciwieństwie do triangulacji Delaunaya, punkty na które ma zostać nałożona siatka meshu muszą być w regularnych odstępach od siebie. W związku z tym, chcąc wykorzystać tę metodę trzeba dokonać interpolacji punktów, w celu przejścia z nieregularnych przestrzeni pomiędzy elementami do równomiernie leżących punktów. Ze względu na schemat działania algorytmu, jest on niezwykle szybki, ponieważ pozwala on na uniknięcie części obliczeń. Algorytm dzieli przestrzeń na regularne sześciany, a następnie sprawdza przecięcia punktów ze ścianami sześcianu. Przy regularnej siatce punktów istnieje $2^8$ takich przecięć, ponieważ sześcian ma 8 krawędzi, na każdej z nich może leżeć punkt lub nie. Po uwzględnieniu symetrii obrotowej sześcianu liczbę kombinacji można zredukować do zaledwie 15. Zostały one przedstawione na rysunku ~\ref{fig:marchingCubesCut}.
\begin{figure}[H]
  \centering
  \includegraphics[scale=0.5]{przeciecia.PNG}
  \caption{Możliwe przecięcia sześcianu przez punkty \cite{lorensen1987marching}}   
  \label{fig:marchingCubesCut}
\end{figure}
Przecięcia punktów z sześcianami tworzą trójkąty, które później zostaną dodane do triangulacji. Należy zwrócić uwagę na to, że gęstość meshu na obiekcie poddanym triangulacji bezpośrednio zależy od długośći boku sześcianu. Im jest on mniejszy, tym więcej będzie przecięć punktów z krawędziami, a co za tym idzie, więcej utworzonych trójkątów triangulacyjnych. Występują odmiany tej metody wprowadzające zmienną długość boku danego sześcianu \cite{shu1995adaptive}. Adaptacyjna metoda maszerujących sześcianów bada krzywiznę powierzchni wewnątrz sześcianów za pomocą wyznaczania wektorów normalnych utworzonych dla danych trójkątów. Jeśli wektory odchylone są w podobną stronę, oznacza to, że powierzchnia jest dostatecznie gładka. W przeciwnym wypadku sześcian dzielony jest na kolejne 8 sześcianów i algorytm jest powtarzany. 

\subsection{Ball pivoting algorithm}


