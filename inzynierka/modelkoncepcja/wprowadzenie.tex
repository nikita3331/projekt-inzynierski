\section{Wprowadzenie}
Skanery 3D swoim wyglądem i zasadą działania przystosowane są do różnych zastosowań. W celu skanowania dużych obiektów takich jak budynki korzysta się z drogich skanerów wielkoformatowych. Za to jeśli użytkownik chce zeskanować mały obiekt, możliwe jest dokonanie tego urządzeniem ręcznym. W dalszym ciągu pracy zostanie przedstawiony autorski projekt skanera 3D razem z opisem jego zasady działania. Przedstawione zostaną podstawy teoretyczne przekształcenia danych z kamery głębi do chmury punktów. Ukazane zostały również algorytmy służące do utworzenia meshu na podstawie chmury punktów.
