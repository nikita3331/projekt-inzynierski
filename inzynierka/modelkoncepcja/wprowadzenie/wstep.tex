Skaner jest urządzeniem służącym do rejestracji oraz konwersji do postaci cyfrowej płaskich obrazów na przykład dokumentów lub zdjęć. Jednakże zwykły skaner nie umożliwia rejestracji trójwymiarowych obiektów. Do tego zadania używany jest skaner 3D. Początki skanerów 3D datuje się na lata 60 XX wieku. Do uzyskania modeli używano wówczas oświetlenia oraz kamer, które rejestrowały odkształcenia cienia mierzonego obiektu \cite{ebrahim20153d}. Niestety, wymagania dotyczące nakładu pracy nie były proporcjonalne do otrzymanych wyników. W połowie lat 80-tych komputery zyskały na popularności, a narzędzia pomiarowe stały się dokładniejsze w efekcie czego postanowiono użyć sondy stykowej. Mierzono odkształcenie sondy po zetknięciu się z obiektem wskutek czego można było wyznaczyć położenie punktów na płaszczyźnie obiektu w innym układzie współrzędnych \cite{abdel20113d}. Jej użycie pozwoliło na znaczne zwiększenie dokładności pomiarów, jednak prędkość ich wykonywania była powolna. Wobec tego zaistniała zauważalna potrzeba opracowania metody optycznej, która umożliwiłaby mierzenie obiektów z większą prędkością. Miałoby to na celu również pomiar elastycznych przedmiotów, które dotychczas nie były mierzalne ze względu na użyte technologie. Istnieje wiele różnych podejść do trójwymiarowych skanerów, każde z nich ma zastosowanie w określonej dziedzinie. Poniżej przedstawiono najważniejsze z nich.