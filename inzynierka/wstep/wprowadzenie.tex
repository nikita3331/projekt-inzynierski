
Kluczowym czynnikiem który wpłynął na rozwój technologii skanerów trójwymiarowych jest druk 3D. Jego historia rozpoczyna się w 1984 kiedy to Charles Hull zaprojektował oraz stworzył urządzenie korzystające z procesu zwanym stereolitografią \cite{gokhare2017review}. Jest to proces polegający na utwardzaniu materiału za pomocą lasera. Światło padając na powierzchnię płynnego fotopolimeru tworzy jego utwardzoną warstwę. Następnie platforma jest obniżana, a proces jest powtarzany aż do momentu utworzenia całego obiektu. Korzystając z druku 3D jest możliwe szybkie utworzenie prototypu z plastiku. Pozwala to na sprawdzenie na przykład właściwości aerodynamicznych modelu 3D bez konieczności wytwarzania go z docelowych materiałów, co znacząco zmniejsza koszty projektowania takich obiektów \cite{bassett20153d}. Do utworzenia komputerowych modeli 3D korzysta się z różnych metod projektowania. Jedną z nich jest trójwymiarowy skan rzeczywistego obiektu, a następnie przeniesienie go do wirtualnego systemu komputerowego. Istnieje wiele rozwiązań na rynku pozwalających na utworzenie precyzyjnych trójwymiarowych modeli. Jednym z takich programów jest Blender, otwarte oprogramowanie dzięki któremu użytkownik jest w stanie utworzyć elementy 3D jak również je animować. Poprzez wykorzystanie tego narzędzia użytkownicy są w stanie nawet wykonywać filmy animowane \cite{BlenderMovies}. Drugim programem jest Maya. Jest to o wiele bardziej zaawansowane oprogramowanie do modelowania 3D. Jest ono wykorzystywane w branży filmowej. Niestety większość ogólnodostępnych programów jest płatna oraz ma mało możliwości konfiguracji poszczególnych parametrów związanych z rekonstrukcją kształtu obiektów. W niniejszej pracy zostanie zaprezentowany sposób na wierne utworzenie obiektu 3D ze skanów zrobionych za pomocą kamery Intel RealSense D435i.