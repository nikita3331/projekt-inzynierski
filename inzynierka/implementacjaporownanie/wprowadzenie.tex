W bieżącym rozdziale przedstawione zostaną sposoby użycia dwóch metod rekonstrukcji powierzchni z chmury punktów. Omówione zostaną charakterystyki poszczególnych algorytmów oraz rezultaty dzięki nim uzyskane. Ukazana zostanie implementacja wybranego z nich.
W celu uzyskania trójwymiarowych modeli na podstawie skanów rzeczywistych obiektów, przetestowano dwie metody generacji meshu. Pierwszą z nich jest algorytm toczącej się kuli. Kolejnym algorytmem jest trójwymiarowa triangulacja Delaunay'a wraz ze sposobami jej optymalizacji.

