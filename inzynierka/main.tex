% \documentclass[10pt,twoside]{article}
\documentclass{pginz}

\usepackage[utf8]{inputenc}
\usepackage{nicefrac}       % compact symbols for 1/2, etc.
\usepackage{microtype}      % microtypography
\usepackage{ragged2e}
\justifying
\usepackage{float}
\renewcommand{\figurename}{Rysunek}
\usepackage{natbib}
\usepackage{graphicx}
\usepackage{amsmath}
\usepackage{adjustbox}
\usepackage{polski}
\usepackage{gensymb}
\setcitestyle{square}
\usepackage{setspace}
\usepackage{pdfpages}
\usepackage{titlesec}
\usepackage{svg}



\providecommand{\keywordspl}[1]
{
  \small	
  \textbf{\textit{Słowa kluczowe:}} #1
} 
\providecommand{\keywordseng}[1]
{
  \small	
  \textbf{\textit{Keywords:}} #1 
}
\providecommand{\dnauki}[1]
{
  \small	
  \textbf{\textit{Dziedzina nauki i techniki, zgodnie z wymogami OECD:}} #1
}


\begin{document}

\includepdf[pages=-]{tytolowanowa.pdf}
\includepdf[pages=-]{oswiadczenie.pdf}


\setcounter{page}{3}

 
\section*{STRESZCZENIE}
Celem niniejszej pracy dyplomowej było stworzenie skanera 3D oraz systemu wizualizacji utworzonych modeli rzeczywistych obiektów. Do budowy urządzenia wykorzystano kamerę głębi firmy Intel o nazwie RealSense D435i. W pracy został przedstawiony sposób budowy skanera 3D, jego kalibracji oraz algorytmy służące do przetwarzania otrzymanych danych pomiarowych w celu uzyskania wirtualnych modeli. W celu łatwiejszej obsługi programu został utworzony interfejs graficzny zawierający najważniejsze parametry wizualizacji i obróbki danych. Na koniec dane są eksportowane do modeli w formacie obsługiwanym przez program Blender.

\keywordspl{Skaner 3D ,Intel RealSense, Python, Kamera RGBD}

\dnauki{Nauki inżynieryjne i techniczne, Systemy automatyzacji i kontroli }

\section*{ABSTRACT}
The aim of this thesis was to create a 3D scanner and a system for visualization of created models based on real objects. In the work is presented how to build a 3D scanner, its calibration and algorithms used to process the obtained measurement data to obtain virtual models. In order to make the program easier to use, a graphic interface was created containing the most important parameters of visualization and data processing. Finally, the data are exported to the models in a format supported by the Blender program.

\keywordseng{3D Scanner, Intel RealSense, Python, RGBD Camera}
\newpage
\tableofcontents


\newpage

\chapter{WYKAZ WAŻNIEJSZYCH OZNACZEŃ I SKRÓTÓW}

\section{Kamera RGBD}
Kamera głębi, oprócz wykonywania zdjęć RGB potrafi ona również dokonać pomiaru odległości od obiektów i nanieść te informację na powierzchnię poszczególnych pikseli obrazu.

\section{LIDAR}
Ang.(Light Detection And Ranging) urządzenie służące do dokładnego pomiaru odległości. Działaniem przypomina funkcjonowanie radaru, lecz korzysta z odliczania czasu przelotu światła lasera, a nie mikrofal.

\section{Blender}
Oprogramowanie służące do modelowania trójwymiarowego.Posiada szereg funkcji do animacji obiektów, generacji tekstur oraz importowania i eksportowania gotowych modeli.
\section{Maya}
Program komputerowy, umożliwiający generację zaawansowanych modeli 3D przeznaczony do zastosowań przemysłowych. W tym programie zostały stworzone filmy takie jak Spiderman, Avatar oraz Up.
\chapter{WPROWADZENIE}
\section{Wprowadzenie}
W bieżącym rozdziale przedstawione zostaną wyniki implementacji dwóch metod rekonstrukcji powierzchni z chmury punktów. Omówione zostaną charakterystyki poszczególnych algorytmów oraz rezultaty dzięki nim uzyskane. Ukazana zostanie implementacja wybranego z nich.
W celu uzyskania trójwymiarowych modeli na podstawie skanów rzeczywistych obiektów, przetestowano dwie metody generacji meshu. Pierwszą z nich jest ball pivoting algorithm. Kolejnym algorytmem będzie trójwymiarowa triangulacja Delaunay'a.


\section{Cele i założenia}
Celem niniejszej pracy jest zaprojektowanie skanera 3D korzystającego z metody triangulacji laserowej  oraz wyeksportowanie modeli do programu Blender przy zastosowaniu kamery 3D.
[Założenia odnośnie odległości z jakiej skanujemy.Porównanie różnych wyników w zależności od odległości. Wypełnienie skanu teksturami.Czas obróbki.]
\section{Zawartość pracy}
Pierwszy rozdział opisuje cele oraz założenia pracy.Dokonano gruntownej analizy problemu, który zostanie rozwiązany w dalszym ciągu pracy.

W drugim rozdziale wykonano przegląd istniejących metod mających na celu generację trójwymiarowych obiektów na podstawie danych z kamery głębi. Dokonano porównania pomiędzy dostępnymi na rynku skanerami 3D bazującymi na różnych technologiach pomiarowych. Wymieniono ich parametry techniczne. Zobrazowano w jakich warunkach dana metoda pomiarowa powinna zostać wykorzystana. Ukazane zostały również technologie jakimi posługiwano się w przeszłości do generacji trójwymiarowych modeli. Na koniec przedstawione zostały zastosowania współczesnych skanerów 3D.

W kolejnym rozdziale przedstawiony jest model oraz konstrukcja skanera 3D. Wyjaśniono metody służące do przetworzenia danych uzyskanych z kamery głębi w chmurę punktów. Przedstawiono koncepcje istniejących rozwiązań służących do rekonstrukcji powierzchni oraz kształtu obiektów z chmury punktów.

Czwarty rozdział przedstawia metody analizy oraz obróbki danych, które mają posłużyć do cyfrowej implementacji rzeczywistych obiektów zarejestrowanych przez kamerę RGBD. Przedstawiono opisy zastosowanych algorytmów oraz kolejność ich wykonywania na podstawie autorskiego programu w języku Python. Skupiono się również na różnych metodach optymalizacji algorytmów. Poddano analizie pod względem dokładności rezultaty pomiarów w porównaniu do rzeczywistych wartości mierzonych.

Ostatni rozdział zajmuje się podsumowaniem wykonanej pracy oraz otrzymanych wyników. Poruszane są w nim możliwości udoskonalenia urządzenia. Przytoczone zostały również wady istniejącej metody obróbki danych oraz generacji modeli. Dokonano porównania autorskiej metody tworzenia obiektów trójwymiarowych oraz programów dostępnych na rynku.



\section{Wprowadzenie}
W bieżącym rozdziale przedstawione zostaną wyniki implementacji dwóch metod rekonstrukcji powierzchni z chmury punktów. Omówione zostaną charakterystyki poszczególnych algorytmów oraz rezultaty dzięki nim uzyskane. Ukazana zostanie implementacja wybranego z nich.
W celu uzyskania trójwymiarowych modeli na podstawie skanów rzeczywistych obiektów, przetestowano dwie metody generacji meshu. Pierwszą z nich jest ball pivoting algorithm. Kolejnym algorytmem będzie trójwymiarowa triangulacja Delaunay'a.


\chapter{Model skanera oraz koncepcja działania}

\section{Wprowadzenie}
W bieżącym rozdziale przedstawione zostaną wyniki implementacji dwóch metod rekonstrukcji powierzchni z chmury punktów. Omówione zostaną charakterystyki poszczególnych algorytmów oraz rezultaty dzięki nim uzyskane. Ukazana zostanie implementacja wybranego z nich.
W celu uzyskania trójwymiarowych modeli na podstawie skanów rzeczywistych obiektów, przetestowano dwie metody generacji meshu. Pierwszą z nich jest ball pivoting algorithm. Kolejnym algorytmem będzie trójwymiarowa triangulacja Delaunay'a.


\section{Model i konstrukcja skanera 3D}
W celu wykonania dokładnych modeli trójwymiarowych został utworzony skaner 3D na podstawie autorskiego projektu. W skład zestawu wchodzi kamera Intel RealSense D435i oraz platforma obrotowa. Wybór sensora od firmy Intel nie był przypadkowy. Posiada on szereg wbudowanych funkcji, takich jak łatwa możliwość kalibracji oraz nastaw odpowiednich parametrów wykrywania głębi. Jego rozdzielczość oraz dostępna liczba klatek na sekundę sprawia, że akwizycja danych jest o wiele dokładniejsza. Podczas budowy skanera dokonano porównania możliwości dwóch kamer trójwymiarowych: Intel RealSense D435i oraz Orbbec Astra Mini MX6000. W tabeli ~\ref{tab:intelvsorbbec} dokonano przeglądu najważniejszych charakterystyk obu tych skanerów, zestawiono między innymi rozdzielczości głębi oraz kąt przechwytywania obrazu.

\begin{table}[H]
\begin{center}

\caption{\label{tab:intelvsorbbec}Porównanie charakterystyk kamer Intel RealSense D435i oraz Orbbec Astra Mini MX6000 \cite{OrbbecAstraMiniSheet} \cite{IntelRealSenseSheet}.}
\centerline{
\begin{tabular}{ |c| c|c| }
 \hline
 {\small Kamera} & {\small Astra Mini} & {\small RealSense D435i}\\ 
 \hline
 {\small Dokładność} & {\small $\pm$ 1-3mm na 1 m} & {\small < \text{2\%}  na 2 m}\\ 
  \hline
   {\small FOV} & {\small 60 \degree H x 49.5 \degree V} & {\small 87 \degree H x 58 \degree V}\\ 
  \hline
 {\small Rozdzielczość RGB } & {\small 640 px x 480 px } & {\small 1920 px x 1080 px}   \\  
  \hline
   {\small Rozdzielczość głębi } & {\small 640 px x 480 px } & {\small 1280 px x 720 px }   \\  
  \hline
     {\small FPS } & {\small 30} & {\small 90}   \\  
  \hline
   {\small  Długość fali lasera } & {\small 830 nm} & {\small 850 nm}  \\  
  \hline
\end{tabular}
}
\end{center}
\end{table}
Z powyższych charakterystyk wynika, że kamera firmy Intel jest dokładniejsza oraz lepiej spełni zadanie wiernego odwzorowania modelu 3D. Ponadto oprogramowanie dostarczane przez firmę Intel o nazwie RealSense Viewer pozwala na łatwą obsługę tego urządzenia. Umożliwia ono podgląd obrazu z kamery zarówno w 2D jak i w 3D. W programie dostępne są ustawienia, umożliwiające poprawną regulację parametrów rejestracji obrazu, takie jak moc lasera, wartość graniczną wykrywanej głębi oraz ekspozycję. Wszystkie te aspekty znacząco usprawniają proces kalibracji, produktywność oraz wpływają na poprawę dokładności generowanych obrazów.
Konstrukcja zbudowanego skanera została zaprezentowana na rysunku ~\ref{fig:konstrukcjaModelu}. 
\begin{figure}[H]
  \centering
    \includesvg[scale=0.75]{modelSkanera.svg}
  \caption{Schemat budowy autorskiego skanera 3D.}   
  \label{fig:konstrukcjaModelu}
\end{figure}
Na powyższym rysunku można dostrzec dwa kluczowe elementy wchodzące w skład skanera. Platforma ruchoma napędzana silnikiem elektrycznym zapewnia stałą prędkość kątową obrotu tacki. Dzięki temu wyznaczanie położenia obiektu w przestrzeni jest dokładne. Wykonane zostały testy platformy napędzanej silnikiem elektrycznym oraz poruszanej ręcznie. Z wytworzonych w ten sposób modeli wynika jasno, iż stała prędkość kątowa obiektu jest kluczowa do poprawnego przekształcenia modelu. Kolejnym elementem wykorzystanym przy budowie skanera jest kamera RGBD. Wykonuje ona zdjęcia kolorowe oraz głębi z określoną częstotliwością oraz zapisuje je do pliku, w celu późniejszej ich obróbki. Ze względu na wykorzystanie platformy o stałej prędkości kątowej, dokonane zostało porównanie wpływu FPS na wygląd ostatecznego modelu. Zmiana liczby klatek na sekundę wpływa bezpośrednio na rozdzielczość kątową wykonanych zdjęć, gdy ta liczba jest większa, gęstość chmury punktów również się zwiększa.\\
\indent Zasada funkcjonowania skanera została przedstawiona poniżej:
\begin{enumerate}
    \item Mierzona jest dokładna odległość obiektywu kamery od środka tacki.
    \item Obiekt umieszczany jest na obrotowej tacce.
    \item Dokonuje się kalibracji tak ustawionego elementu, tak by stopień wypełnienia punktów był jak najdokładniejszy.
    \item Tacka zostaje uruchomiona z prędkością 0.1 $\frac{rad}{s}$.
    \item Uruchomiony zostaje zapis obrazu głębi oraz RGB z kamery.
    \item Gdy tacka wykona pełen obrót, nagrywanie oraz tacka zostają zatrzymane. 
\end{enumerate}

Wysokość obiektu jest mierzona na podstawie danych z kamery RGBD. Znając odległość kamery od obiektu, można wyznaczyć jego wysokość korzystając ze wzoru:

\begin{equation}
    \begin{aligned}
        H_{m}=D\cdot \frac{h_{pix}}{h_{sens}}\cdot V_{FOV}
    \end{aligned}
\label{equ:wysokoscRealPix}
\end{equation}
W powyższym wzorze $H_{m}$ jest rzeczywistą wysokością obiektu, D określa odległość kamery od obiektu w metrach, zaś $V_{FOV}$ oznacza pionowy kąt widzenia kamery. Wysokość soczewki kamery została oznaczona we wzorze jako $h_{sens}$, a $h_{pix}$ jest wysokością obiektu w pikselach. 

W celu wyznaczenia rzeczywistej wysokości obiektu w zależności od odległości i jego wysokości w pikselach, należy znać pionowy kąt widzenia kamery oraz wysokość soczewki. Informacje o kącie widzenia skanera znajdują się w dokumentacji urządzenia Intel RealSense D435i, natomiast wysokość soczewki należy wyznaczyć empirycznie \cite{IntelRealSenseSheet}. W tym celu, dokonano pomiaru wysokości obiektu w pikselach na obrazie z kamery oraz jego odległość od obiektywu. Badanie powtórzono 8 razy w celu uzyskania dokładnej aproksymacji. Znając odległość oraz rzeczywistą wysokość obiektu, po przekształceniu wzoru można uzyskać wartość $h_{sens}$. Po uśrednieniu wyników ze wszystkich pomiarów przeprowadzono badanie jakości estymacji wysokości. Wykres rzeczywistej wysokości obiektu oraz jej przybliżenia znajduje się na rysunku ~\ref{fig:wysokoscOdleglosc}.
\begin{figure}[H]
  \centering
    \includegraphics[scale=0.55]{wysokoscPredykcja.png}
  \caption{Porównanie rzeczywistej wysokości z jej estymacją.}   
  \label{fig:wysokoscOdleglosc}
\end{figure}
Na powyższym rysunku można zauważyć, że aproksymacja wysokości daje dobre rezultaty. W zakresie od 20 cm do 66 cm, maksymalny błąd względny wyniósł 4.2\%. Otrzymane rezultaty są zadowalające do poprawnej wizualizacji rzeczywistego obiektu.

\section{Przejście do chmury punktów}
Po rejestracji pełnego obrotu obiektu na platformie, uzyskany plik należy odpowiednio przetworzyć. Na podstawie danych w nim zapisanych należy przejść z płaszczyzny dwuwymiarowej do chmury punktów. Poniżej opisany został szereg czynności, które należy wykonać w celu otrzymania docelowego rezultatu. Przedstawiono procedury, prowadzące do poprawnej reprezentacji punktów w przestrzeni trójwymiarowej.
\subsection{Wstępna obróbka danych.}
Pierwszym krokiem użytej metody jest wstępna obróbka danych. Odczytywane są dane zapisane w pliku .bag z kamery RGBD. Następnie, wybierana jest kolumna obrazu która posłuży do ekstrakcji danych z kamery głębi. Filtracja w ten sposób danych ma na celu zwiększenie wydajności kodu. Dzieje się tak, ponieważ do późniejszych algorytmów będzie wykorzystywana tylko jedna kolumna z klatki obrazu, na podstawie której dokonana zostanie ekstrakcja informacji o głębi.

\subsection{Przekształcenie danych z kamery RGBD do współrzędnych 3D.}
Następnym etapem schematu jest przejście z dwuwymiarowego układu współrzędnych kamery do trójwymiarowego układu obiektu. W tym celu zostanie użyte poniższe przekształcenie, wynikiem którego będzie macierz czterowymiarowych punktów. W skład punktu wchodzą trzy współrzędne odpowiadające pozycji w przestrzeni oraz wartość RGB danego punktu symbolizująca jego kolor.

\begin{equation}
    \begin{aligned}
            & D_{\beta}=\begin{bmatrix}d_{0} & \dots & d_{max} \end{bmatrix}  \\
            & RGB_{\beta}=\begin{bmatrix}rgb_{0} & \dots & rgb_{max} \end{bmatrix}  \\
            & X_{\beta}=cos(\beta)(R-D_{\beta})  \\
            & Y_{\beta}=sin(\beta)(R-D_{\beta})  \\
          & Z=\begin{bmatrix} 0 & \dots & H_{max} \end{bmatrix}  \\
          & w_{\beta}^n=\begin{bmatrix} X_{\beta}[n] & Y_{\beta}[n] & Z[n] & RGB_{\beta}[n]\end{bmatrix}  \\
    \end{aligned}
    \label{equ:chmuraPunktow}
\end{equation}

$\beta$ jest kątem obrotu tacki. R jest odległością obiektywu kamery od środka osi obrotu tacki. $D_{\beta}$ to macierz zmierzonych odległości punktów na płaszczyźnie obiektu od kamery dla danego kąta obrotu $\beta$. $X_{\beta}$,$Y_{\beta}$ są macierzami współrzędnych punktów na osi X,Y dla nago kąta $\beta$. $H_{max}$ to wysokość obiektu. Z to macierz współrzędnych punktów na osi Z. Wypełniona jest ona punktami z zakresu od 0 do $H_{max}$. $w_{\beta}^n$ to współrzędne n-tego punktu w płaszczyźnie XYZ dla danego kąta obrotu $\beta$.

\subsection{Normalizacja otrzymanych punktów.}
Kamera RGBD jest zaawansowanym technicznie urządzeniem. Pomimo kalibracji kamery, w przechwyconych obrazach mogą występować niedoskonałości. Przejawiają się one w złych pozycjach punktów względem pozostałych sąsiadów. Jest to spowodowane przekłamanymi odczytami głębi przez sensor kamery i jest to nieuniknione. Wynika to z konstrukcji skanera oraz odbić światła lasera od powierzchni obiektu. W celu zmniejszenia wpływu błędnych pomiarów na ostateczny wygląd obiektu należy dokonać normalizacji punktów. Normalizacja sprowadza się do zbadania, czy owy punkt leży w odległości od środka podobnej do jego sąsiadów. W tym celu została wyznaczona średnia odległość punktów od osi przechodzącej przez środek obrotu tacki, która ma współrzędne (0,0,H). Warto zauważyć, że wartość współrzędnej Z nie pływa na wyznaczanie odległości, ponieważ utworzona ona została na podstawie równomiernego rozkładu wartości wysokości danego obiektu. Następnie, znając wartość średniej odległości punktów od środka układu współrzędnych, można empirycznie dobrać wartość graniczną odległości powyżej której punkty będą poddane interpolacji.

\subsection{Interpolacja punktów}
Interpolacja jest procesem aproksymacji współrzędnych punktów w miejscach, w których wystąpiły przekłamania. Jest wiele różnych metod interpolacji, które mają za zadanie jak najwierniej oddać wartość punktu w nieznanym miejscu. Głównymi metodami są metoda najbliższych sąsiadów oraz interpolacja wielomianowa.

\subsubsection{Metoda najbliższych sąsiadów}
Interpolacja korzystająca z metody najbliższych sąsiadów jest często stosowanym algorytmem do rekonstrukcji danych w nieznanym miejscu. Wykorzystywana jest na przykład przy powiększaniu zdjęć do uzyskania lepszej rozdzielczości \cite{han2013comparison}. Zasada działania tej metody została przedstawiona poniżej.
\begin{enumerate}
    \item Wybierany jest punkt P, którego wartość ma zostać wyznaczona.
    \item Punkty w chmurze nie są rozłożone równomiernie, dlatego też wyznaczenie sąsiadów sprowadza się do wyznaczenia odległości wszystkich pozostałych punktów od wybranego.
    \item Dla każdego punktu w chmurze mierzona jest jego odległość od wybranego punktu. Współrzędne punktu są trójwymiarowe, więc odległość od sąsiada $D_{n}$ może zostać obliczona za pomocą wzoru:
    \begin{equation}
             D_{n}=\sqrt{(x_{p}-x_{n})^2+(y_{p}-y_{n})^2+(z_{p}-z_{n})^2}    \\
    \end{equation}
    
$x_{p},y_{p},z_{p}$ są współrzędnymi punktu, a $x_{n},y_{n},z_{n}$ są współrzędnymi n-tego sąsiada.
    \item Ze zbioru wyznaczonych w ten sposób odległości wyznaczana jest najmniejsza. Punktowi zostaje przypisana wartość najbliższego do niego punktu.
\end{enumerate}
\subsubsection{Interpolacja wielomianowa}
Interpolacja wielomianowa jest szeroko stosowanym zagadnieniem w matematyce oraz fizyce. Jej szczególnymi odmianami są interpolacja liniowa oraz kwadratowa. Polega ona na dopasowaniu funkcji na przykład liniowej,kwadratowej lub dowolnego innego stopnia wielomianu do zbioru punktów. Znając funkcję przechodzącą przez wszystkie punkty ze zbioru, można określić jaka będzie jej wartość w punkcie którego wartość była dotychczas nieznana. Równania przedstawiające tę metodę zostały przedstawione poniżej.
\begin{equation}
    \begin{aligned}
            &X=\begin{bmatrix}
                    x_{0}^0 &\dots & x_{0}^n\\
                     \vdots  & \ddots & \vdots \\
                    x_{n}^0 &\dots & x_{n}^n
                \end{bmatrix}\\
            &A=\begin{bmatrix}
                    a_{0}\\
                      \vdots \\
                    a_{n}
                \end{bmatrix}\\
            &Y=\begin{bmatrix}
                y_{0}\\
                  \vdots \\
                y_{n}
            \end{bmatrix}\\
            &Y=X \cdot A\\
            &A=X^{-1} \cdot Y\\
            & W(x)=a_{0}+a_{1}\cdot x +a_{2}\cdot x^2+\ldots +a_{n}\cdot x^n
    \end{aligned}
    \label{equ:wielomianowaEqu}
\end{equation}

X jest macierzą współrzędnych x ze zbioru dostępnych punktów. A jest macierzą współczynników wielomianu W(x). Y jest macierzą współrzędnych y ze zbioru dostępnych punktów. W(x) jest wielomianem przechodzącym przez wszystkie punkty ze zbioru.

Znając wielomian interpolacyjny można określić współrzędne poszukiwanego punktu, który znajdował się poza zbiorem dostępnych punktów, jego współrzędne to P($x_{p}$,$W(x_{p})$)

\section{Rekonstrukcja powierzchni}
Wszystkie powyższe metody pozwalały na określenie chmury punktów danego obiektu. W celu wyeksportowania gotowego modelu, potrzebna jest również znajomość całej powierzchni danego przedmiotu. W tym pomocne okazują się różne metody rekonstrukcji powierzchni. Jest to kluczowy element działania algorytmu wirtualizacji rzeczywistych obiektów do postaci modeli 3D. Powstało wiele prac naukowych na temat różnych metod rekonstrukcji powierzchni, poniżej zostały omówione najważniejsze z nich.
\subsection{Trójwymiarowa triangulacja Delaunaya}
Triangulacja powierzchni metodą Delaunaya jest jedną z popularniejszych metod nakładania siatki na nieregularnie rozprowadzone punkty. Polega ona na połączeniu punktów w nieprzecinające się trójkąty, dzięki czemu możliwe jest nałożenie na nie koloru oraz utworzenie meshu na danym obiekcie. Triangulację Delaunaya można wykonać dla dowolnej N-wymiarowej płaszczyzny, w danym przykładzie zostanie opisana metoda dla trójwymiarowej chmury punktów \cite{cignoni1998dewall}. Schemat działania tej metody został opisany poniżej. 
\begin{enumerate}
    \item Tworzony jest czworościan zawierający wszystkie punkty ze zbioru.
    \item Tworzona jest lista czworościanów do usunięcia oraz czworościanów do pozostawienia.
    \item Dla każdego punktu sprawdzane jest czy leży on wewnątrz sfery opisanej na czworościanie do pozostawienia. Jeśli leży,to tworzone są 4 czworościany z wierzchołków starego oraz z nowego punktu.
    \Item Na końcu początkowy czworościan wewnątrz którego znajdowały się wszystkie punkty zostaje usunięty. Skasowane zostają również wszystkie czworościany mające z nim wspólny wierzchołek.
\end{enumerate}
Zaletą tej metody jest to, że jest ona inkrementacyjna. Punkty do triangulacji są dodawane po kolei,a co za tym idzie, nie ma potrzeby tworzenia nowej siatki po dodaniu pojedynczego punktu. Wystarczy jedynie wykonać jedną fazę algorytmu i dopisać utworzone wierzchołki czworościanów do listy.
Modyfikacją tej metody jest triangulacja DeWall zaproponowana przez włoskich naukowców w 1997 roku \cite{cignoni1998dewall}. Jej nazwa pochodzi od Delaunay i ściana (ang. \textit{Wall}). Pojęcie ściany oznacza w tym wypadku rekurencyjne dzielenie zbioru punktów na dwa podobne do siebie obszary, następnie równoległe utworzenie dwóch niezależnych od siebie triangulacji Delaunaya. Dzięki podzieleniu zbioru punktów na niezależne od siebie obszary, możliwe staje się wykorzystanie wielowątkowości procesora do obliczeń równoległych. To znacząco przyspiesza pracę algorytmu.
\subsection{Algorytm maszerujących sześcianów}
Algorytm maszerujących sześcianów jest metodą typu dziel i podbijaj zaproponowaną w 1987 roku przez amerykańskich naukowców w Nowym Yorku \cite{lorensen1987marching}. W przeciwieństwie do triangulacji Delaunaya, punkty na które ma zostać nałożona siatka meshu muszą być w regularnych odstępach od siebie. W związku z tym, chcąc wykorzystać tę metodę trzeba dokonać interpolacji punktów, w celu przejścia z nieregularnych przestrzeni pomiędzy elementami do równomiernie leżących punktów. Ze względu na schemat działania algorytmu, jest on niezwykle szybki, ponieważ pozwala on na uniknięcie części obliczeń. Algorytm dzieli przestrzeń na regularne sześciany, a następnie sprawdza przecięcia punktów ze ścianami sześcianu. Przy regularnej siatce punktów istnieje $2^8$ takich przecięć, ponieważ sześcian ma 8 krawędzi, na każdej z nich może leżeć punkt lub nie. Po uwzględnieniu symetrii obrotowej sześcianu liczbę kombinacji można zredukować do zaledwie 15. Zostały one przedstawione na rysunku ~\ref{fig:marchingCubesCut}.
\begin{figure}[H]
  \centering
  \includegraphics[scale=0.5]{przeciecia.PNG}
  \caption{Możliwe przecięcia sześcianu przez punkty \cite{lorensen1987marching}}   
  \label{fig:marchingCubesCut}
\end{figure}
Przecięcia punktów z sześcianami tworzą trójkąty, które później zostaną dodane do triangulacji. Należy zwrócić uwagę na to, że gęstość meshu na obiekcie poddanym triangulacji bezpośrednio zależy od długośći boku sześcianu. Im jest on mniejszy, tym więcej będzie przecięć punktów z krawędziami, a co za tym idzie, więcej utworzonych trójkątów triangulacyjnych. Występują odmiany tej metody wprowadzające zmienną długość boku danego sześcianu \cite{shu1995adaptive}. Adaptacyjna metoda maszerujących sześcianów bada krzywiznę powierzchni wewnątrz sześcianów za pomocą wyznaczania wektorów normalnych utworzonych dla danych trójkątów. Jeśli wektory odchylone są w podobną stronę, oznacza to, że powierzchnia jest dostatecznie gładka. W przeciwnym wypadku sześcian dzielony jest na kolejne 8 sześcianów i algorytm jest powtarzany. 

\subsection{Ball pivoting algorithm}







\renewcommand{\listtablename}{Spis tabel}
\renewcommand{\listfigurename}{Spis rysunków}
\addcontentsline{toc}{chapter}{Spis tabel}
\addcontentsline{toc}{chapter}{Spis rysunków}
\listoffigures
\listoftables


\addcontentsline{toc}{chapter}{Bibliografia}
\bibliography{references}
\bibliographystyle{ieeetr}
\end{document}
